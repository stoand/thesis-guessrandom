\documentclass{article}
\usepackage[utf8]{inputenc}
\usepackage[english]{babel}
\usepackage[margin=0.5in]{geometry}
\usepackage[document]{ragged2e}
\usepackage{multicol}

\begin{document}

    \begin{center}
    \end{center}
    
    \addvspace{20mm}
        
    \begin{center}
        \huge Novel use of a Functional HDL to Simplify Development of an RNG Brute-Force Algorithm
    \end{center}
    
    \begin{center}
    \end{center}
       
    \begin{center}
        \large Andreas Stocker
    \end{center}
    
    \begin{center}
        \small \emph {University of Nicosia}
    \end{center}

    \addvspace{15mm}

    \begin{multicols}{2}

    \section*{Abstract}

    In this paper we go over the implementation details of an algorithm that brute forces
    the state of a random number generator (RNG). Recovering the internal state of an RNG can allow
    future outputs to be predicted, possibly compromising the security of a system.
    We do so using a hardware description language whose syntax
    resembles that of the Haskell language.
    The purpose of this paper is to assess the practicality of using this functional-style approach in
    designing calculation-heavy algorithms for FPGAs.
    First, we go over the basics of using this high-level hardware description named "Clash".
    We do this in a way that even programmers of more mainstream languages like Java should be able to follow.
    Finally, we go over the difficulties that were overcome as well as any special techniques involved in
    the implementation of the brute force algorithm.

    \section{Introduction}

    todo intro
    
    \section{Related Work}

    Todo - add this

    THIS COMES NEAR THE END

    
    \end{multicols}

    \break
    \section*{References}

    \begin{enumerate}

    \item Chandrasekaran, Shrutisagar, and Abbes Amira. "High performance FPGA implementation of the Mersenne Twister." 4th IEEE International Symposium on Electronic Design, Test and Applications (delta 2008). IEEE, 2008.

    \end{enumerate}
    
\end{document}
