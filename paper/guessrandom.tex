\documentclass{article}
\usepackage[utf8]{inputenc}
\usepackage[english]{babel}
\usepackage[margin=0.7in]{geometry}
\usepackage[document]{ragged2e}
\usepackage{multicol}

\begin{document}

    \begin{center}
    \end{center}
    
    \addvspace{20mm}
        
    \begin{center}
        \huge Novel use of a Functional HDL to Simplify Development of an RNG Brute-Force Algorithm
    \end{center}
    
    \begin{center}
    \end{center}
       
    \begin{center}
        \large Andreas Stocker
    \end{center}
    
    \begin{center}
        \small \emph {University of Nicosia}
    \end{center}

    \addvspace{15mm}

    \begin{multicols}{2}

    \section*{Abstract}

    In this paper we go over the implementation details of an algorithm that brute forces
    the state of a random number generator (RNG). Recovering the internal state of an RNG can allow
    future outputs to be predicted, possibly compromising the security of a system.
    We do so using a hardware description language whose syntax
    resembles that of the Haskell language.
    The purpose of this paper is to assess the practicality of using this functional-style approach in
    designing calculation-heavy algorithms for FPGAs.
    First, we go over the basics of using this high-level hardware description named "Clash".
    We do this in a way that even programmers of more mainstream languages like Java should be able to follow.
    Finally, we go over the difficulties that were overcome as well as any special techniques involved in
    the implementation of the brute force algorithm.

    \section{Introduction}

    Today's CPUs make use of the Von Neumann architecture. The Von Neumann architecture dictates the use
    of seperate CPU and memory modules. Furthermore, it dictates that a programs instructions be stored
    in memory.

    This works well for many of today's workloads. This current system has the benefits of decades of optimization
    and feature development.

    However, certain types of computations are bottlenecked by this approach. This is especially true for
    highly parallel calculations.

    GPUs address this need for parallelism but they are still limited by a design that prioritizes
    certain types of calculations.

    Sometimes programmers require a greater degree of control over hardware. Arguably the greatest degree
    of control developers can have is by developing their own ASICs. Modern CPUs and GPUs most
    other silicon chips are themselves ASICs. These ASICs are incredibly expensive to produce however,
    with prices often ranging from the millions into the billions.

    Therefore ASICs are well out of reach for most programmers to use for their personal projects, or even for
    medium sized companies.

    This is where FPGAs enter the picture. Less powerful FPGAs are cheap and widely available. FPGAs themselves
    are often used to prototype ASICs. However they can also be used to implement algorithms where precise control
    over circuitry is needed.

    FPGAs allow circuits to be dynamically programmed using lookup tables (LUTs). The more LUTs an FPGA
    has the more complex the circuits it can be configured for. These circuits can be configured for
    extreme parallelism. This is because unlike conventional CPUs, FGPAs are not limited by something like a core count,
    they simply have wires on a circuit.

    Despite having these strong advantages, one of the greatest disadvantages FPGAs suffer from is complexity of development.
    Since developers are working on a far lower level than even assembly language, the complexity
    of programming FPGAs is often far higher. Programmers familiar with high level languages
    like Java will not be able to use a lot of the knowledge they aquired and will need to learn electrical engineering
    concepts to create advanced FPGA projects.
    
    \section{Related Work}

    Todo - add this

    THIS COMES NEAR THE END

    
    \end{multicols}

    \break
    \section*{References}

    \begin{enumerate}

    \item Chandrasekaran, Shrutisagar, and Abbes Amira. "High performance FPGA implementation of the Mersenne Twister." 4th IEEE International Symposium on Electronic Design, Test and Applications (delta 2008). IEEE, 2008.

    \end{enumerate}
    
\end{document}
