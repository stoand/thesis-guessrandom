\documentclass{article}
\usepackage[utf8]{inputenc}
\usepackage[english]{babel}
\usepackage[margin=1.9in]{geometry}
\usepackage[document]{ragged2e}
\usepackage{listings}
\usepackage{setspace}

\linespread{1.3}

\begin{document}

    \begin{center}
    \end{center}
    
    \addvspace{20mm}
        
    \begin{center}
        \huge Novel use of a Functional HDL to Simplify Development of an RNG Brute-Force Algorithm
    \end{center}
    
    \begin{center}
    \end{center}
       
    \begin{center}
        \large Andreas Stocker
    \end{center}
    
    \begin{center}
        \small \emph {University of Nicosia}
    \end{center}

    \addvspace{15mm}

    \section*{Abstract}

    Todo add abstract....
    
    TODO - add results/conclusions after project is done

    \section{Introduction}

    TODO

    \section{Introduction to FPGAs}

    The sophisticated FPGAs of today are the product numerous incremental improvements
    over the years [3]. One such step in the evolution are Programmable Read Only Memories
    also known as PROMs. These proms were used to implement logic gates. There are also
    different varieties of PROMs where some of them can only be programmed once and
    others which could be reprogrammed multiple times. PROMs had a drawback in that
    sequential logic could not be completely encapsulated within a PROM and would need
    to be added to a circuit as separate components. Another glaring drawback of PROMs was
    their lack of speed.

    Programmable Logic Arrays also known as PLAs made significat improvements over PROMs [3].
    Namely, PLAs were generally must faster that PROMs. They could also support a far
    larger number of inputs. Though one drawback was that number of combinations of
    logic elements was slightly more constrained than that of of PROMs.
    
    \section{FPGA Architecture}

    TODO

    \section{Related Work}

    Todo - add this

    THIS COMES NEAR THE END

    \break
    \section*{References}

    \begin{enumerate}

    \item Chandrasekaran, Shrutisagar, and Abbes Amira. "High performance FPGA implementation of the Mersenne Twister." 4th IEEE International Symposium on Electronic Design, Test and Applications (delta 2008). IEEE, 2008.

    \end{enumerate}
    
\end{document}
