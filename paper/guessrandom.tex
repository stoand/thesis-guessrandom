\documentclass{article}
\usepackage[utf8]{inputenc}
\usepackage[english]{babel}
\usepackage[margin=1.9in]{geometry}
\usepackage[document]{ragged2e}
\usepackage{listings}
\usepackage{setspace}

\linespread{1.3}

\begin{document}

    \begin{center}
    \end{center}
    
    \addvspace{20mm}
        
    \begin{center}
        \huge Novel use of a Functional HDL to Simplify Development of an RNG Brute-Force Algorithm
    \end{center}
    
    \begin{center}
    \end{center}
       
    \begin{center}
        \large Andreas Stocker
    \end{center}
    
    \begin{center}
        \small \emph {University of Nicosia}
    \end{center}

    \addvspace{15mm}

    \section*{Abstract}

    Todo add abstract....
    
    TODO - add results/conclusions after project is done

    \section{Introduction}

    TODO

    \break

    \section{Introduction to FPGAs}

    The sophisticated FPGAs of today are the product numerous incremental improvements
    over the years [3]. One such step in the evolution are Programmable Read Only Memories
    also known as PROMs. These proms were used to implement logic gates. There are also
    different varieties of PROMs where some of them can only be programmed once and
    others which could be reprogrammed multiple times. PROMs had a drawback in that
    sequential logic could not be completely encapsulated within a PROM and would need
    to be added to a circuit as separate components. Another glaring drawback of PROMs was
    their lack of speed.

    Programmable Logic Arrays also known as PLAs made significat improvements over PROMs [3].
    Namely, PLAs were generally must faster that PROMs. They could also support a far
    larger number of inputs. Though one drawback was that number of combinations of
    logic elements was slightly more constrained than that of of PROMs.

    Programmable Array Logic (not to be confused with Programmable Logic Arrays) were the next
    step in the evolution that would culminate in the FPGAs of today [3]. Programmable Array Logic
    also known as PALS added support for clock elements as well as flip flops. They
    were much more sophisticated in their support for expressing sequential logic. Futhermore,
    they had the benefit of great performance.

    The FPGA was designed with the goal of accomplishing the same computations as ASICs but
    with the added benefit of reprogrammability. Because of this, FPGAs are frequently
    used to emulate ASICs, as well as to act as temporariy stand-ins while ASICs
    are still being produced. FPGAs as they are today have benefitted greatly
    from advances in CMOS design that were initially made with improving CPUs in mind.
    However, it was precisely because CPUs were becoming so efficient that custom hardware
    lose a good amount of popularity [6]. It became easier for companies to simply using
    general purpose CPUs instead of investing in tailor-made hardware level designs.
    One example of such a use-case where general purpose CPUs won is databases [6].
    This was in the late 70's and researchers were trying to create a "databse-machine".
    It was specifically tailored to run database queries on low-level hardware.

    When it comes to the benefits of using FPGAs verses CPUs there are several factors to keep in mind.
    Image and signal processing are two applications where FPGAs shine [6].
    Partially this is because FPGAs can generally offer a greater level of determinism
    that CPUs. In an FPGA latecy for some tasks can often not only be lower, but the latecy
    and also be predictable. This is crucial for applications where real-time
    responsiveness is key. Another place where FPGAs shine is parallelism [6].
    CPUs main unit of parallelism is their cores. FPGAs however, have a far more granular
    unit of parallelism which is their logic blocks. This allows for orders of magnitude
    more potential parallelism over CPUs. These factors make FPGAs ideal candidates
    for high-throughput low-latecy applications.

    When in comes to solving real-world problems, FPGAs are quickly moving out of niche,
    highly-specialed projects and are becoming more common in commondity setups [6].
    The main areas where FPGAs are gaining popularity are both in the networking sector
    as well as the graphics processing sector. One reason for this is that the amount
    of data generated in the world is growing rapidly. For this reason, the raw processing
    power of FPGAs is expected to become indispensable for many more companies in the future.

    One concrete use case for FPGAs is database co-processing [6]. This is because
    streaming databases are required to process data with a low latecy even under heavy
    load. Ironically, this is one area that is similar to the "database-machine" style
    projects of the 70's. So perhaps FPGA designers will run into the same sort of
    problems that "database-machine" researchers encountered [6] all those years ago.
    Undianiably though, there is a large room for improvement in this sector.
    One reason FPGAs are still somewhat niche is the fact that FPGA design is not
    very accessible to "mainstream" programmers. This is because "mainstream" programmers
    are familiar with the so-called Von Neumann architecture. The Von Neumann architecture
    assumes instructions are executed from memory and run in a given order. Neither of these
    factors apply when programming FPGAs.

    There are several projects which aim to overcome the reduced ease of use of FPGAs
    for mainstream programmers. The "Kiwi" and "Liquid Metal" projects aim to do exactly
    this [6]. The goal is that FPGAs be used with general purpose languages.
    However the general sentiment in the FPGA developer community seems to be that these
    projects that allow mainstream languages be used with FPGAs are not developed
    enough to be used for mission-critical applications.
    It could be that at the most fundamental level FPGAs are incompatible with languages
    designed for Von Neumann runtimes.
    Another way in which mainstream appeal for FPGAs can be improved is by
    providing developers with out of the box IP that developers can setup on FPGAs to interface
    with projects running on CPUs programmed in mainstream languages like C.

    Another point of comparison between FPGAs and CPUs is clock speed [6]. FPGAs need to operate
    at significatly lower clock speeds than CPUs. Though the amount of work done by FPGAs
    in a single clock cycle can sometimes be orders of magnitude greater in FPGAs. This benefits
    FPGAs by reducing the amount of power consumed by the device relative to a CPU. One downside
    is that the FPGA may be able to do less computations overall due to this slower speed.

    Yet another noteworthy point of comparison between FPGAs and CPUs is their memory model.
    CPUs with their Von Neumann architecture, generally have memory on a separate chip
    than the CPU. Though even if memory is technically on the same die as the CPU, like in Apple's
    M1 architecture, CPUs will always suffer from what is called the "Von Neumann bottleneck".
    Also termed the "memory wall" occurs because the entire CPU can only access one area
    of memory at once. One thing that mitigates this disadvantage is the ability to
    access a continous block on memory at once. This advantage does not exist if the memory
    is non in a continous block however. Memory is significatly different in FPGAs though.
    FPGAs have flip-flop registers as well as block ram [6]. Spare lookup tables can
    also be reprogrammed at runtime and used as additional memory. So in general, FPGA memory
    boasts far superior locality compared to CPU memory. It therefore has much better
    throughput, as well as lower latecy compared to the CPU.
    One nifty feature that exists because of the FPGA's memory model is known as
    "content addressible memory" (CAM). Content addressible memory can be used
    to implement a key value store with a constant lookup time.

    One thing modern CPUs have going for them when it comes to competing with FPGAs are
    SIMD operations [6]. This allows for parallel operations that mimic those of FPGAs.
    This means an operation (like addition for example) can be performed on two fixed
    sized arrays of values.

    \subsection{Open Source and FPGAs}

    When it comes to the development workflow, there is one major place where mainstream CPU programming
    is ahead of the tools used to develop for FPGAs. That is the arena of open source tools.
    Open source tools have grown to a point today where virtually all major tech companies
    either use open source or even contribute to it. These include Google, Apple and Amazon
    just to name a few.
    There are a number of benefits to having an open source ecosystem. The first major
    benefit is accessibility. If anyone can install a tool for free this is great
    for students and anyone who wants to learn about the project.
    Another benefit of open source is reliability. If a vendor simply stops updating a closed
    source project then everyone dependant an it is in serious trouble if bugs
    crop up and there is no one to fix them.
    Another factor is security. The more experienced people read and understand a piece of
    code the more likely it is that issues can be discovered.
    Open source is also great if the consumers of a project want to add their own features.
    There are some upsides to closed source software though. Closed source software with
    licensing fees means that a company can pay experienced developers to work on the project full
    time. This means the project is not dependant on people working on it on their free time.
    Of course, this does not apply to all open source projects as some projects are popular
    enough to have paid full time developers working on them.
    Also, when in comes to security through obscurity, the argument can be made that
    if attacker do not have access to the source code of a project, they can't find
    exploits for it as easily.

    The current state of open source tools for FPGA designs is as follows:
    project Icestorm provides decently feature-rich tools for a small number of FPGA
    models. Not only does Icestorm provide most features needed throughout the FPGA
    design workflow, it is even superious to proprietary tools in some ways. Icestorm
    does not suffer from the massive bloat and slowness of conventional FPGA tools.
    Not only are the tools included in Icestorm free from bloat, they can also be
    a lot faster some times.

    The downside of relying on the open source Icestorm project is that it largely depends
    on people investing their free time, and if these people lose interest in a particular
    tool that is part of the project, the entire workflow that uses this project
    may become untenable.

    \break
        
    \section{FPGA Architecture}

    An FPGAs internal architecture affects it's speed, area efficiency, and power consumption [1].
    FPGAs compete with ASICs in that they both allow a digital circuit to be created.
    They also have several advantages over ASICs because they can be reprogrammed in seconds
    and cost orders of magnitude less. ASICs can cost millions to produce but they do have
    their own benefits. FPGAs pay the price for their easy reconfigurability in area, delay,
    and power usage. ASICs are 20-35 more compact than FPGAs and require 10 times less power.
    This is because the FPGAs programmable routing circuitry causes overhead. Nevertheless,
    despite these downsides, FPGAs are far more viable for small to medium sized companies
    that can't spend millions on ASICs. ASICs cost millions because of three main factors.
    First, ASICs need expensive software in order for their circuits to be designed.
    Second, ASICs need a mask so their circuitry can be entched into silicon.
    The cost of this mask can be reduced by multiple different ASICs sharing a mask.
    Finally, hiring engineers to design a complex ASIC over multiple years is also quite
    expensive. These engineers cannot make even a small mistake in their design
    because that would ruin a mask which costs millions to produce.

    FPGAs suffer from no such complications because they can instantly be reprogrammed.
    In fact, FPGAs are often used by ASIC engineers to prototype and test their designs.

    Then it comes to their internal architecture, FPGAs consist of a variety of different
    components [1]. These include logic, memory, and multiplier blocks. All around these
    blocks is a programmable routing fabric that allows blocks to be configured to both
    communicate with each other as well as with the outside world through inputs and outputs.

    When it comes to memory, modern FPGAs use either flash, static memory, or anti-fuses [1].
    SRAM, a type of static memory, is very common in modern FPGAs like from the manufacturers
    Xilinx, Lattice and Altera. These SRAM cells perform two main roles in their FPGA's
    architectures. First, a majority of SRAM is used to configure interconnect signals.
    Most of the left over SRAM is used to persist information in lookup tables also known as
    LUTs.

    \subsection{Memory Variants}

    SRAM is used frequencly in modern FPGAs because it has a number of advantages [1]. SRAM
    does not have a limit to how many times it can be reprogrammed. This is unlike the
    EPROM of early FPGA which could only be reprogrammed once or a couple hundred times.
    Second, SRAM does not require any special electrical components and can be etched
    in silicon with CMOS techniques. These CMOS techniques allow FPGAs to benefit from
    all the production advancements made for modern CPUs.

    SRAM is not without its drawbacks howerver. Every SRAM cell needs 5-6 transistors
    which is quite a lot. Second, SRAM cannot persist information if a system is powered
    off. This necessitates the need for another system of persistent storage which increases
    the complexity of the system. Persistent storage is often provided by flash or EEPROM.
    Finally, storing a design in a persistent storage system like flash or EEPROM also
    makes it far easier to competitors of a company to reverse engineer a design.
    To mitigate this risk encryption is used.

    An alternative to SRAM is the so called floating gate technology [1]. This is used
    in flash or EEPROM which do not lose the data stored in them when power is lost.
    This flash based setup offers a number of benefits besides persistence.
    Persistence removes the need for a seperate storage mechanism. Persistence also
    makes it possible to run the device immediately after it starts up because there
    is no need for the programming step. This is important for certain use cases.
    Additionally, flash boasts greater area-efficiency compared to SRAM.

    One disadvantage brought by flash is that the floating gate requires that
    charge injection needs to be prevented. Another more serious downside of flash
    is its limited lifespan because can only be reprogrammed a fixed number of times.
    This is in contrast to SRAM which can be reprogrammed a virtually infinite amount of times.
    Depending on the application, this can be a serious disadvantage or a non-issue.
    The "Actel ProASIC3" for example can only be reprogrammed 500 times.
    This definitely could be an issue at least for prototyping where dozens of reprogramming
    cycles can occur in a day. Another very serious issue with flash where SRAM
    is superior is in the production run. One previously mentioned benefit of SRAM
    is that it can be produced using standard CMOS techniques. This is not the case with
    flash. Flash requires specialized components can cannot be created by etching
    silicon like with CMOS.

    In modern FPGAs there is a trend towards using a combination of flash storage and SRAM.
    This has the benefit of allowing infinite reconfigurability but this comes at a price
    of greater area overhead.

    The final type of FPGA configuration systems that are common today are anti-fuses [1].
    The most glaring difference between anti fuses and flash and SRAM is that the former
    can only be programmed once. This of course makes it unsuitable for many FPGAs applications
    like ASIC prototyping where multiple configuration runs are a neccesity.
    The name anti-fuse comes from the idea that the FPGA fabric consists of a number of fuses
    that can be selectively "blown". Though a more accurate description of this process
    would be the idea a of connecting these "fuses". This connection occurs when high voltage
    is sent through the "fuse". Most modern anti-fuses are mental-to-metal based.

    While the fact that anti-fuses cannot be reprogrammed after the initial configuration,
    they do come with the benefit of greater area efficiency [1]. In metal-to-metal
    anti-fuses this is because no silicon area is expended to allow configurable connectivity.
    One thing that does consume a significant amount of area however is the need for
    programming transistors.

    Another benefit of anti-fuses is the ability to include a greater number of switches
    because of lower on resistances and parasitic capacitence. Also,
    another obvious benefit that is shared with flash base FPGAs is the ability to
    work instantly when powered on since the fabric cannot change once programmed.

    Yet another benefit shared with flash based systems is that the design
    of the FPGA is harder to reverse engineer because it is not stored anywhere except
    the logic configuration itself. The security of the FPGA also benefits from the
    fact that because the FPGA can only be programmed once a malicious actor
    cannot reprogram the FPGA if given access to the device.

    When it comes to the manufacturing process, however, anti fuse FPGAs cannot
    benefit from using a CMOS silicon etching process [1]. Even worse,
    the fabrication process of anti fuse FPGAs is beginning to run into scaling problems
    and is generally lagging behind the sophistication of modern CMOS.

    To summarize, all three of these memory systems have their own pros and cons and
    which memory system will be used needs to be considered in light of the application
    and its constraints and goals. SRAM (as well as the flash SRAM hybrid approach) can
    be considered to be dominant howerver. This can be attributed to its compatibility
    with the sophisticated CMOS manufacturing process.
    
    \subsection{Logic Blocks}

    The basic unit of computation and storage within an FPGA is known as a logic block [1].
    The smallest a single logic block can be is to take the form of a single transistor.
    Though logic blocks can be far larger than that and even take the form of an entire
    CPU in theory. The size of a logic block, however, is something that needs to be carefully
    balanced for maxium efficiency to be achieved.

    If a logic block is too small for example, it will suffer from area inefficiency because
    it will increase the need for programmable routing. Furthermore, performance, and
    power consumption will also be negatively impacted.

    On the other extreme, logic blocks that encapsulate a large amount of logic may
    be performant, but they nullify the benefits an FPGA provides in the form of programmable
    logic. In effect, the circuit will become less and less of an FPGA, and more of
    a standard ASIC.

    Selecting logic blocks according to their size and functionality is a key consideration
    that FPGA creators (as in people creating the FGPA itself, not a design for it) need
    to consider. There are three main factors that need to be considered when selecting logic
    block type and instance count: these are area, speed and power.

    Logic blocks in modern FGPAs fall into two main categories: normal logic blocks,
    and blocks that perform specialized computations, like addition for example.
    What specialized computations will be included need to be carefully considered
    because unused specialized logic will inevitably lead to wasted area.

    Despite FPGAs being considered a completely general purpose device,
    creators of the FPGAs will in practice need to consider what kind of designs
    will run on a target FPGA [1].

    There is one main tradeoff that needs to be considered here:
    how much functionality is encapsulated into every logic block and how many
    logic blocks total are needed.
    Increasing the functionality in a single logic block reduces to total
    number of logic blocks needed (up to a diminishing point that is).
    Though as the logic block size increases, the number of wires connecting
    it in the routing increases.

    Most modern industrial FPGAs use a heirarchical approach that uses
    clusters of LUTs and flip flops [1]. This helps control the granularity
    of logic blocks.
    This involves grouping basic logic components together and connecting
    them with a local interconnect.
    This approach is used instead of increasing LUT size.
    The benefit is that the needed routing only grows quadradically instead
    of exponentially.

    To consider how logic block size affects speed several factors need to be considered.
    As stated before, increasing the size of logic blocks decreases the number of them
    that need to be used. Fewer logic blocks being used means that less routing
    logic needs to be employed. This in turn means greater performance
    because the electrical signal needs to travel a shorter distance.

    However, this reduction in delay which is outside the logic blocks is
    accompanied by a larger delay within the logic blocks.

    When it comes to power consumption, the tradeoffs are similar to those
    with area and speed [1]. Reducing the area also reduces the power needed.

    Calculating power, area, and speed is simpler when considering only
    one size of LUT. However, a heterogenous combination of LUTs can sometimes
    be more efficient when considering the different metrics.

    In contrast to LUTs, specific purpose logic is more efficient (if its being used) [1].
    Specific purpose logic is more efficient because internally, it
    has all the benefits of an ASIC and non of the inefficiencies of programmable logic.

    A downside of specific purpose logic is that it its not being used, its
    a definite net negative on all fronts.
    
    \subsection{Routing}

    The purpose of routing in an FGPA is to provide a programmable fabric that connects
    logic blocks and IO ports. Under the hood, routing uses wires which are connected
    by programmable switches [1]. For routing it is important that a large variety
    of circuit configurations are possible. Just like with logic blocks, performance
    and power consumption are in need of consideration when it comes both to designing
    the routing and configuring it to take the shape of a design.

    "Locality" in routing refers to the level of proximity that interconnected components
    exhibit. In general, it can be said that circuits exhibit a high level of locality
    because most components are close. Of course, there is also the need for connecting
    components that are far from each other.

    There are also special types of signals which need to be available globally
    in the FPGA. [1] These include clocks and resets. These kinds of signals get
    special treatment in the form of a dedicated interconnect system. These
    interconnect systems are designed to minimize skew. This means that the
    variation or noise added to the signal as it travels is minimized.

    The main type of routing that designers need to keep in mind is the so called
    general purpose routing. This is different from the global interconnect routing
    mentioned previously.
    
    One type of general purpose routing is called global routing. Global routing
    considers the properties of a routing design on a higher level while not paying attention
    to the details. general purpose routing mainly considers the locations of routing
    channels, the number of wires in a given channel, and how various channels communicate.

    In contrast to general purpose routing, detailed routing takes into consideration
    wire length, switch counts and how wires and logic block pins connect.

    When it comes to the global architecture of an FPGAs routing there are two main styles.
    The first global routing style is the hierarchical style. In this style,
    logic blocks are separated into groups. Logic blocks contained in the same group
    are connected by wire segments. Logic blocks in different groups are connected
    by wires that traverse multiple levels of routing segments.
    The further routing channels get from logic blocks, the more wires they tend to contain.

    There are several factors that need to be considered when using a hierarchical routing
    layout. One benefit of the hierarchical routing layout is that the delay
    between logic blocks is often more predictable. Performance can also be better for
    certain types of designs.
    However, if there is a mismatch between the length of a design's wires and the
    heirarchical distribution, problems may arise. Furthermore, moving between levels
    of the heirarchy can cause a significant delay. These are some of the reasons
    why modern FPGAs largely do not make use of the hierarchical routing system.

    What modern FPGAs use now for routing is the so called island style of routing.
    Island style routing arranges logic blocks in a two dimensional array and surrounds
    them with routing resources. One key decision that the designers of the FPGA itself
    need to consider here is the width of a channel.
    There are several benefits to using the island style of routing and it is likely
    that these contributed to making it the most popular style of routing in use today.

    Since a logic block has access to a varienty of wire lengths, the most efficient length
    wire can be selected. Furthermore, minimum routing delay between logic blocks is trivial
    to estimate.

    Within the island style design, there are a variety of switch block designs that
    can be used. These include the Wilton, disjoint and universal switch block designs.
    
    \subsection{Challenges of using FPGAs}

    Despite all of the improvemnts made in the field of FPGAs, there are a number of challenges
    that need to be overcome so further development can be made [1]. As CMOS processes
    have continued to become more sophisticated, several issues have become more prominent.
    As silicon chips continue to decrease in size, so called soft errors become more and more
    of a problem. A soft error occurs when ionizing radiation corrupts some data in a circuit.
    This sort of corruption does not only occur in FPGAs. Its also known problem in
    regular computer RAM. This is actually the reason why enterprise grade RAM uses
    ECC technology to detect this sort of corruption.

    The source of the radiation itself can come from both radioactive packaging and
    from cosmic radiation.

    There are a number of ways in which soft errors can be reduced in an FPGA.
    The first mitigation is at the circuit and technology level.
    One useful circuit level change is selecting an optimal memory supply voltage.
    Another technique is to add metal capacitors to memory nodes which
    decreases their sensitivity to radiation.


    A higher level mitigation is at the system level. Here, the designers of the FPGA have added builtin checks
    to find and correct corruption [1].
    One system level mitigation periodically checks the state of configuration memory
    with its correct value. If an error is found, the FPGA will need to be reprogrammed.
    A "don't care" flag is set for resources that are not currently in use since
    corruption in used resources will not affect the FPGAs functionality.
    Another, far more expensive form of mitigation, is the so called triple modular
    redundancy. Effectively, the FPGA design is replicated three full times.
    The circuitry will then vote on what output values are correct.
    This extreme of an approach would likely not work for most everyday use cases.
    Though for things like medical equipment, or for any place that handles money this
    could be useful.
    It is also noteworthy that routing is the cause for a majority of soft errors.

    When it comes to user visible memory this can also be a source of soft errors,
    the flip flops used here are actually not as vulnerable to corruption as SRAM.
    This is because SRAM is far smaller in size than flip flops. Neverthess,
    like in enterprise grade RAM, error correction can also be used here.
    It should be noted howeve, that the error correction itself is not perfect. Some
    errors will still evade detection.

    Adding such error correction comes with a cost. This is because additional data will
    need to be stored and because of the need for special encoding and decoding circuits.

    This cost can be reduced in modern FPGAs by the use of hard circuits that
    do memory encoding. Again however, like with ECC, these error correction mechanisms
    only reduce errors and do not eliminate them entirely.

    Also, a distinction needs to be made between mere error detection and full error correction.
    Since the user's design has full control over memory, the onus falls upon the user
    to hanlde errors. This is less desireable than handling errors at the FPGAs system level.
    This is because handling errors at the system level would only need to be done once,
    at the time of the FPGAs creation. These user level checks will add additional complexity
    to each and every design that uses an FPGA.
    Again, it is also up to the user to decide on the level of error correction depending
    on their application. Some applications might not cause serious consequences if
    soft errors do occur.

    There is another source of irregularities that can occur in FPGAs. The silicon etching
    process produces products that exhibit a certain degree of variations.
    Not all of these variations cause outright errors. Some of these variations simply
    cause decreased performance or increased power consumption. During manufacturing,
    as is done with CPUs, every FPGA needs to be individually tested.

    This is another area where shrinking CMOS sizes have caused an increase in complications.
    The smaller the circuts, the greater the variations in performance and power consumption
    become.
    Not only do FPGA have variation between each other as a result of this manufacturing process,
    there also exists variation within the FPGA itself. This form of variation
    is far more problematic. This is because entire FPGAs can easily be discarded,
    whereas malformed areas within an FPGA cannot be removed. These malformed areas
    mean that the clock speed of the entire FPGA needs to be reduced. For
    a 22nm process, this variation in performance can be as much as 22.4\% [1].

    Another type of problem that affects FPGA functionality are manufactoring defects.
    These are different from the process variations mentioned above because an FGPA
    with process various may still be able to works, whereas an FPGA with defects
    may be totally unusable. In the chip manufacturing industry there is a concept of
    "yield". Yield refers the the percentage of chips in a manufacturing run that
    are usable. Chips that are not part of the yield need to be discarded. These
    discarded, unusable chips need to be factored into manufacturing costs.

    Yet again, as CMOS processes improve and chips become smaller, complications arise.
    Smaller, more complex CMOS processes generally result in manufacturing defects and therefor
    significatly lower yields.

    A possible workaround for manufactoring defects, is to still use FPGAs that have manufacturing
    defects and simply not to make use of defective areas.
    The downside of this workaround is that testing a large number of FPGA chips for compatability
    with a single design may not be economical.

    Instead of simply ignoring defective areas, another approach is to build in redundancy
    from the start. This sort of approach is common in memory devices [1]. The level of
    redundancy can either be fine grained or coarse grained. Coarse grained redundancy
    adds entire rows or columns of tiles. This course grained approach has the downside
    that a significant amount of additional routing logic becomes necessary. This
    additional routing in turn has a negative impact on performance and power consumption.
    Challenges also arise when coarse grained redundancy interacts with heterogenous
    block layouts.
    A more fine grained approach merely adds additional switches. These additional switches
    are used to bypass defective routing. One benefit of fine grained redundancy is that
    these additional switches can even be used if no defects are present.
    One challenge of the fine grained approach is that a very detailed map
    of defects is needed so place and route tools can work around them.
    This detailed map adds significant complexity to the design workflow and tooling.

    Coarse grained redundancy also has the benefit of being able to handle defects in
    routing as well logic. This is in constrast to fine grained redundancy which
    cannot deal with logic block errors. Though the fine grained approach is better
    and handling interconnect errors.

    
    \subsection{FPGA Architecture Conclusion}

    The last couple of sections illustrate that FPGAs are a complex and multifacted
    technology. The people creating the FPGAs themselves as well as the developers
    of FPGA designs needs to consider a wide range of different factors. This massive
    complexity does have an upside however. It allows solutions that are very well
    tailored to specific use cases. 
    
    
    \section{FPGA Design Workflow}

    The design flow deals with the steps needed to eventually program an FPGA
    to solve a certain problem [3]. The design flow of FPGAs is actually quite
    similar to the design flow used for CLPDs and ASICs.
    In the previous section the creation of the actual FPGA itself was discussed,
    as well as programming the FPGA. This seciton will only cover the programming
    part as well as the steps leading up to it.
    The main distinction between creating an FPGA and programming for it is that
    the FPGA creators can chose what components end up on the FPGA while programmers
    merely make use of the components provided on their target FPGA.

    \subsection{The Specification}

    While some may view a formal specification as optional, a serious project
    will most likely suffer greatly if its creators forgo a real specification [3].
    Furthermore, in a team setting, a specification helps communicate to each team
    member not only what their own role is but also how their section fits into the
    larger project. The prevents multiple engineers from designing pieces that
    do not work together.

    A specification covers a number of different implementation details.
    One very useful tool to have in a specification is block diagrams [3].
    These block diagrams can both show how a design fits into a larger system,
    as well as the internal design of the FPGA. An internal block diagram shows
    how major components within the FPGA exist and are connected to one another.

    Another important piece of content is a description of what input and outputs
    will connect the FPGA to the outsie world. Another important concept that
    needs to be documented when it comes to I/O are timing estimates.

    The timing estimate for input pins covers setup and hold times.
    Output pins on the other hand need to have propagation times detailed.

    Finally there is a global clock cycle time that needs to be clearly visible.

    After the intial draft of a specification is complete, it becomes important
    to prevent the specification from growing stale by always keeping it updated.
    Its simply not possible to know every aspect of the design beforehand so
    its important that when the team decides to make changes that these changes
    are also committed to the specification.

    Part of the work of drafting a specification deals with selecting what major components
    the design will have. Weighing the positives and negatives of various FPGA products
    is one of the major parts of this process. There are numerous factors that need to be
    considered. These include the cost of the parts, their performance, their compatability
    with each other, among other factor. Another dominant factor may be
    the level of experience members of the team have with different vendors and product lines.

    Another incredibly important step in the designing phase is selecting a design
    entry method [3]. Small and less complex designs may simply use a schematic entry.
    In a schematic entry all of the routing is done manually.
    This approach has the upside of designers getting a very fine degree of control over
    how the FPGA is configured.
    There is however, a very serious downside to such and approach and that is that it
    is not viable for any even somewhat sophisticated design.

    The two technologies that are by far the most common for design entry are the languages
    Verilog and VHDL. These languages are highly portable in that they abstract
    over the low level differences that exist between FGPA models.
    They are also far more readable, flexible and expressive. These languages
    allow for what is called "synthesis" in which a software program is read
    and a low level logic gate design is generated from it.

    While these languages may share some similarities with mainstrea programming languages
    like Java, such as a shared concept of a "for" loop, they are in fact quite different.
    For starters, these languages make a distiction between logic connections that
    always exist, as well as synchronous changes that occur every clock cycle.

    Another large difference between HDLs and high level programming languages is that
    they are mostly built on the concept of wires and what connections occur between these
    wires. There have been some experimental research projects like "Kiwi" [1] that
    attempted to allow compilation from standard high level languages like C to HDLs.
    These however, have not gained mainstream traction in industry as well as for any serious
    projects in general. It seems there is just too much of a mismatch between
    these systems that prevents elegant abstractions from being created.

    After an HDL has been selected, the next step in the design process is to choose a
    synthesis tool. The purpose of the synthesis tool is to generate a logic gate
    from an HDL. Various different synthesis tools exist, but one tool that
    is particularly notable is the tool that is included in the Icestorm project.
    The Icestorm project is notable because it is completely open source, something
    which is quite rare in the hardware design industry in general.

    Following the selection of a synthesis tool, the next step in the process
    is designing a chip [3].

    \subsection{Simulation and Testing}

    While chip design is ongoing, it is important to be also meanwhile running simulations.
    Even smaller components should be tested using simulation. A lack of simulations
    can become a serious issue due to the rise of bugs that can be hard to trace if
    the design becomes a huge black box. Another great benefit of simulation is that
    in addition to getting a design running correctly, simulations are very useful
    as a form of documentation. If a developer begins working a section they know nothing
    about, simulations can serve as a reference of what the design is supposed to do.
    Simulations can even be useful to the initial developer of a component if they design
    something and come back to it weeks or months later.
    Another key role of simulation is to speed up the workflow. Since simulations
    are running regular CPUs and not on FPGAs themselves, they allow for HDLs to
    be recompiled far fasters. This is critical to the workflow, since depending on
    the complexity of the design, spending minutes or even hours resynthesizing for
    every little change can seriously hamper developer productivity.
    Not only is simlation faster, it also allows developers to have complete knowledge
    of the state of a system. This means that developers can inspect the states
    of wires that are internal to the design instead of just being able to inspect the
    outputs.

    However, after developers are satisfied with the design and it has been well tested
    in simulation, a synthesis step indeed becomes necessary.
    During synthesis, the high level hardware description language needs to be converted
    to an FPGA configuration. Part of this process involves performing place and route
    calculations.

    Another key step in converting HDL code into a real FPGA design is timing analysis.
    Timing analysis ensures that the electrical signals that travel through the FPGA
    have enough time to propagate. Without proper timing analysis, serious errors
    can occur due to the FPGA running on too high a clock speed. Once timing analysis
    is complete, designers will be given a maximum clock speed.
    Some FPGA tools, like the Icestorm project, will prevent the FPGA from being
    programmed with too high a clock speed.
    
    After a design is generated and the maximum clock speed has been found,
    it is finally time to program the FPGA.

    Once the design is programmed onto the the FPGA, manual tests can be performed
    to ensure the FPGA outputs the correct values for a given group of input values.

    Or if the FPGA is part of a larger circuit, the testing step involves ensuring
    that the complete system works as intended. If any problems are discovered,
    the workflow repeats itself and additional changes to the HDL as well as
    the simulation become necessary. Also, known bugs and other issues that were found
    should be added to the specification.

    When everything works as intended, the system will be ready to be put into production.
    A burn in test is used to ensure that a system keeps working over a long amount of time.
    However, even if a system was perfectly designed, electrical and mechanical issues
    can still arise and that is why burn in tests are useful.

    \subsection{Potential Design Problems}

    One serious design problem that may occur is race conditions [3].
    A race condition can occur if two signals which affect a given output
    can be triggered in a non deterministic order. Race conditions are dangerous
    because they depend on miniscule delays caused by variations in interal timing.
    Even a tiny change in voltage or temperatue can cause a change in the output.

    This can become a very serous problem because chips that were working perfectly
    may suddenly exhibit erroneous output after weeks, months, or even years of
    reliable operation. One way in which race conditions can be dealt with
    is to introduce a delay by converting an asynchronous operation to a synchronous one.
    However there really isn't a magic fix to eliminate all race conditions since
    they depend on a variety of complex, interwoven states.

    Another serious problems that can arise in an FPGA design is hold time violations [3].
    They are similar to race conditions in that they exist because of a timing problem.
    A hold time violation occurs if data changes at the exact same time as a clock edge.
    This means the resulting value is non-deterministic and can go either way.

    Yet another design problem caused by inconsistent timing is glitches. These occur
    when an output goes high for a very short amount of time. Unlike race conditions
    and hold time violations, eliminating glitches is more straighforward.
    To eliminate a glitch an output can be synchronized by sending it through a flip flop.

    "Metastability" is another design problem. Metastability occurs when a asynchronous
    signal is fed into a synchronous flip flop. In essence, the problem occurs because
    an asynchronous part of the design is interacting with a synchronous part.
    The fix occurs by correctly synchronizing an asynchronous signal to a given clock.
    However, there is no easy fix for metastability. Use of a synchronizer flip flop
    is a partial solution but there is still the danger of a very small chance
    that the flip flop will not resume a valid logic level. The danger of this occuring
    increases with high clock frequences. Some FPGA manufacturers include special
    synchronizer flip flops for exactly the purpose of mitigating this problem.
    Another way to decrease the change a synchronizer flip flop will not work is
    to simply include more of them.

    In general, asynchronous design is prone to a greater number of problems
    than synchronous design. This is because small, hard to debug timing errors
    can arise in asynchronous design.
    In synchronous design the delay is controlled by flip flop that are attached
    to a single clock.
    All of the problems discussed previously are because of asynchronous design
    and can be fixed by using a synchronous style.

    \section{History of Cryptology}

    Cryptology is the study of secret writing [22]. This study of secret writing
    can be broken down into two main disciplines.
    The first of these, "cryptography", concerns itself with techniques
    for creating secret writing. In a way, cryptography represents what could
    be considered the defensive side. It seeks to aid people in
    protecting secrets.
    Cryptanalysis on the other hand, also deals with finding ways of breaking
    crytography. This is the offensive side of this kind of research.

    Throughout the history of cryptology, which spans over 2 millenia,
    a wide range of methods were created with goal of hiding sensitive
    messages.

    \subsection{Ancient Times}

    On of the earliest recorded uses of cryptography starts with the ancient Greeks [22].
    Polybius, a Greek historian, developed a monoalphabetic substitution cipher. It
    was so influencial that the techniques it used influced cryptography two thousand
    years after its creation.
    When creating the cipher, Polybius' goal was initially not even to hide a message.
    He started with the intention of creating a system that could allow signal fires
    to communicate of large distances. In this system, every letter was represented
    by two numbers. It had the downside of having a ciphertext that was twice as long
    as necessary, though some inefficiencies could be expected given that this
    was one a very early variant.

    A few hundred years after Polybius the Roman empire had risen to prominence.
    The famous Julius Caesar himself actually was recorded as having used cryptography
    in his miliary campaigns in Gaul [22].

    Though this is not Caesars only use of cryptography during his lifetime.
    The chipher he used here was fairly straightforward. It takes the letters A-Z
    and randomly reorders them. This is the first time a monoalphabetic substitution
    cipher with a shifted alphabet was used.
    Caesar used this cipher to communicate confidentially with people loyal to him.

    After the decline of Rome, cryptography become less and less used in the western
    world for some time.
    This decline also coincided with downtrend of literacy and general scholarly activity.

    Though in the Arab world in the middle ages there is some recorded use of cryptography.
    For nearly a thousand years, the monoalphabetic substitution cipher was one
    of the strongest ciphers known. Though in the Arab world a ciphe would be developed
    that finally improved upon the status quo.

    During the ninth century AD, in a period also known as the Islamic Golden Age,
    there was a polymath scholar named Al-Kindi. Al-Kindi, being a polymath, was
    well versed in various scientific disciplines which included astronomy, philosophy
    and medicine amoung other fields. He also authored a book intended for
    secretaries in royal counts which showed how secret messages can be used.

    \subsection{The Renaissance}

    As stated before, after the Romans, there was a significant decline in
    virtually all fields of scholarship in the western world, specifically Europe.

    This changed with the start of the Renaissance in the 13th century.

    During this period, there lived a Franciscan monk named Roger Bacon.
    Though he lived in monasteries for most of his life, he did a lot of writing
    and experimenting.
    One of his most significant scholarly achievements was the translation
    of various Arabic texts on science and mathematics.
    
    He did not write a letter on cyprography but instead included various
    cryptographic techiques in a letter for William of Paris.

    Seven techniques were described by him in total.
   
    His philosophy about cryptography was that it was good for keeping secrets
    from the ignorant or uneducated.

    Around this time there was another user of cryptography who was arguably
    far more influencial. This influencial figure was none other than Geoffrey Chaucer.
    Chaucer was a poet who also dabbled in astronomy. He makes use of encryption
    in one of his astronomical books in sections where the usage of some atronomical tools
    is described.
    The cipher Chaucer used was the monoalphabetic substitution cihper which was hundreds
    of years old by this point.

    So far, all the people described here where only mentioned working with cyprography.
    That is, they only dealt with the protection of a secret. In this section the development
    of cryptanalysis, which deals with ways of breaking cryptography, will be explored.

    One of the first cryptoanalytic tools that was created was frequency analysis.
    Frequency analysis allowed ciphers to be broken by examining the frequency of
    encrypted letters. Some letters, such as the letter "a", are simply more common
    in language in general.
    The polymath Al-Kindi discovered this technique. He noted that, in monoalphabetic
    substition cipher, simply substituting one letter for another does nothing
    to obscure the frequency of that letter.

    \subsection{Cryptanalysis}

    This discovery by Al-Kindi was a major step forward in the field of cryptanalysis.

    Another variant of frequency analysis counts not only single letters, but
    pairs of letters. These pairs, also known as "digraphs", are even more powerful
    in breaking monoalphabetic substition ciphers.

    It wasn't until the 16th century however that cryptology began to gain
    mainstream appeal.
    Cryptology started to be used in both the military as well as in commercial applications.
    It was also in this time that a cipher that was far stronger was created.
    This far more advanced cipher would remain virtually unbreakable for hundreds of years.
    It was called the polyalphabetic substition cipher.

    Another important cipher was the biliteral cipher. This cipher was actually
    making use of stenographical techniques. This cipher was created
    by Sir Francis Bacon. Bacon was actually not only a scholar, but also quite
    involved in government. His exploits gained him quite a significant following.

    What was significant about Bacon's biliteral cipher, was that it was resistant to
    freqency analysis. It is likely that Bacon was aware of frequency analysis,
    which had become quite well known during his time, and designed his cipher
    specifically to be unbreakable.

    Where stenography comes into play here is in the encoding of an important message
    inside a fake, meaningless messsage. Furthermore, the meaningless message is
    a couple times longer than the actual message.

    According to some followers of Francis Bacon, Bacon used cyprtographic techniques
    to encode a message in plays. They also argue that Shakespears plays where
    not in fact written by Shakespear, but by Bacon. This is what is known
    as the "Baconian Theory".

    Another significant development in the cryptology world during this time
    was the use of nomenclators. Frequency analysis was a well known weakness
    of monoalphabetic substition at the time.
    The basic idea was the addition of homophones as well as a code book.
    This technique for mitigating frequency analysis did prove quite useful.

    The first nomenclator was created by Gabriele di Lavinde in 1379 in Italy.
    He created this technique for the antipope Clement VII.
    It takes a regular monoalphabetic substition cipher and combines it with a
    small code book.

    Improvements on this techniques that were made later did not use a codebook
    but homophonic substition.

    There are several downsides of using nomenclators though. The main
    downside was that despite frequency analysis becoming more difficult,
    it was still somewhat effective here.
    
    Also, all parties that wanted to encrypt or decrypt a message needed
    to have a codebook. This codebook could be found by an adversary and
    used to break the encryption.

    Despite suffering from all these downsides, nomenclators became popular
    for diplomatic, and to a lesser extent, military applications.
    As the popularity of encryption grew, the demand for cryptanalysis grew
    as well.
    This led to the rise of so called "Black Chambers". These had the purpose
    of breaking cryptography of rival or enemy factions.

    One elite Black Chamber served the Vatican.

    With dedicated code breakers in the form of Black Chambers
    growing in popularity, the arms race
    between code authors and code breakers continued.

    Cryptoanalysis had begun to gain an upper hand over cryptography,
    meaning that it became harder and harder to make unbreakable
    encryption.
    This created a need for stronger cryptography.
    Two main methods were created to strengthen cryptography.

    These were the so called "modern code", as well as the polyalphabetic substitution
    cipher.
    The monoalphabetic substition cipher, the cipher used by Julius Caesar all those
    years ago, could not stand against frequency analysis. This was
    because it simply did not obscure the original text enough
    which made it too easy to observe characteristics about the text
    despite it being encrypted. Something that helped a little bit
    in obsuring the original text was simply the removal of spaces, periods, commas,
    and so on.
    This still is not enough to hide letter frequencies however.
    A better way to obsuring letter frequencies is to use multiple cipher alphabets.
    This means that a letter can be mapped to various different letters.

    \subsection{America at War} 

    A few hundred years later, the American revolution also saw a large
    use of cryptology. This was only in the later parts of the war
    that the Americans made use of cryptography.
    At the start of the war cryptography actually was not used at all
    in communications. This changed when amatuer cryptographers
    that pushed for the use of cryptography to prevent the British
    from intercepting their messages.

    After the Americans began using cryptography, the level of
    cryptographic sophistication that both sides had was about equal.

    A lot of the cryptography on both sides was somewhat experimental however.

    Undoulbtedly though, cryptography played a crucial role in the war.
    This was because it protected important strategic information.
    Despite the significant role cryptography playe in the role,
    neither side had dedicated, full time cryptographers in their employ.
    As the war progressed, both side's military intelligence agencies
    became more and more sophisticated.

    In the American Civil War, significant progress was made in
    both the fields of cyprography as well as cryptanalysis.

    In 1844 the telegraph became more and more popular. Naturally,
    this technology also became adapted for use in the military.
    The first time the telegraph was used in war was actually by
    the British in Crimea in 1853.
    The telegraph was used by both sides in the American Civil War.

    One improvement that came as a consequence of the war,
    was an step up from traditional codes. This was because
    codes were both challenging to use in the theater of war,
    and if one code book was lost, every codebook that contained
    a copy of those codes would have to be replaced.

    Another downside of using codes was that they were not practical for
    the sheer amount of information that was flowing through telegraph
    stations.
    To fix these downsides, so called field ciphers were implemented.

    One instance of cryptanalysis during the war occurred when a
    Confederate soldier named Alexander received a Union cryptogram
    from a captured courier [22].
    The encryption used here was new to Alexander. He was able
    to quickly realize that the encryption hid its secret by simply
    reordering words. He termed this a "jumble". After expending significant
    effort to try to break the jumble, Alexander was not able to break it.

    This jumble turned out to be the Union's main cipher. This cipher was
    created by the telegrapher named Anson Stager. Stager initially started
    with a simple route word transition cipher. Then Stager modified the cipher
    to make it even harder to break. One thing he did was to add "nulls" or "blind words".
    One additional benefit of these nulls was their ability to be used for checking
    the consistency of a message. This was extra useful for telegraph messages which
    could become corrupted during their encoding and decoding processes.
    Stager also encoded some special words through the use of code words.
    Additionally, he added a "commencement word". The purpose of the commencement
    word was to encode certain encryption properties like the size of the rectangle.

    The complexity of ciphers grew a lot during the war. The cipher used
    at the start of the war could fit on a small card. In contrast, the fourth cipher
    that was realeased towards the end of the war was so complex it needed
    a book with dozens of pages.

    When it comes to comparing cipher complexity used by both sides, the Union used
    just one somewhat simple cipher. This is in contrast to the confederates who
    used two completely different ciphers. On of these ciphers was the
    Vigenere cipher. The Vigenere cipher was supposed to be very secure.
    Though somehow the Union was able to break many messages that made use of this cipher.
    Polyalphabetic substitution ciphers were actually quite secure that this time,
    but there was a certain factor that made the Vigenere cipher vulnerable.

    The Vigenere cipher was vulnerable because of the way it was used.
    There were a number of factors that led to the security of this cipher to be
    compromised.
    Their first mistake was that word divisions were not kept hidden but were
    plainly visible on the cryptograms. This made it much easier for the Union's
    cryptanalysis experts to guess words.
    Another serious mistake made by Confederate crytologists was to only
    encrypt parts of messages. This was likely because encryption and decryption
    were quite time intensive.
    While this may seem like a clear problem for security, it could actually
    be argued that this would strengthen the security of the encrypted message.
    This is because the less text cryptanalysis personnel has to work with,
    the hard it should be to guess the original message.
    In practice however, this made messages easier to break because the unencrypted
    message gave context to the encrypted sections.
    Another mistake the Confederates made with their encryption was to use
    only three keys for their highest security cipher.

    Even without the presence of implementation flaws, the Vigenere cipher itself was
    broken in the middle of the nineteenth century. A critical flaw in the cipher
    would be found that created a realiable cryptoanalytic technique for breaking
    the cipher.
    The Cambridge mathematics professor Charles Babbage had a great deal of
    experience in various types of mathematics. He even pioneered several ideas
    that would be used in the computers we have today. Though one could say that
    he had the character flaw of always starting new projects and never finishing anything.
    In 1852 Babbage found a way of breaking polyalphabetic ciphers.

    Without any communication with Babbage, another solution for breaking the polyalphabetic
    cipher was actually discovered independantly. This solution would come
    from a retired Prussian army commander. His name was Friedrich Kasiski and he published
    a book on his techniques for breaking the cipher in 1863.

    Both of these men found the same flaw in the Vigenere cipher independantly.
    The main idea needed to break the cipher made use of the fact that a fatal
    flaw in its implementation was the repetition of the key.

    This techique wasn't perfect however. In short ciphertext it was possible there
    would be no repeated sections. Also, the key could be too long or certain dupliations
    could be mere coincidences.
    One improvement was to have a way of finding the key length. This technique was
    discovered some time later in 1920 by the American William Friedman. Friedman had
    developed numerous techniques while working at the Cipher Department in Illinios.
    Friedman's technique made use of rigorous statistical methods.

    In conclusion, in the twentieth century cryptoanalytic techniques were becoming
    more and more sophisticated. The invention of the telegram also coincided
    with increased use of crytography. With techniques developed by Babbage,
    Kasiski and later Friedman, polyalphabetic ciphers became less and less reliable.
    This set the stage for more sophisticated cryptographic ciphers.

    \subsection{World Wars}

    The first World War saw the introduction of cryptographic systems that
    were quite similar to what is used today [22]. Like telegraphs in the nineteen
    hundreds, the introduction of radio revolutioned communication.
    This also had a large influence on the development of cryptography and cryptoanalytic
    techniques. Army commanders were using radio to send commands to their troops.
    This greatly increased the volume of communication which again affected how
    cryptography would be used because of the large increase of raw ciphertext.
    For the first time, ciphers became so sophisticated that specialized machines
    needed to be built. The first World War also saw the creation of America's
    first dedicated cryptoanalytic agency.

    Cryptographic techniques became so sophisticated that during the course of the war,
    specialized machines became necessary for cryptography. This wasn't only because
    they were more sophisticated. This was also because of the use of radios greatly
    increased the bandwidth that needed to be encoded and decoded. For the majority of
    history, messages were sent through horseback riders which greatly limited the bandwidth
    of messages. Telegraphs increased this bandwidth but there was still a limited amount
    of telegraph stations again limited the amount of information that needed
    to be decrypted and encrypted. Radios were not limited to such telegraph stations
    nor did they require wires to be laid.

    Processing so much encrypted data simply became untenable for manual encryption methods.
    
    Another reason machines were used is becaus electronic equipment became quite a bit
    more sophisticated during this time.

    Yet another trend that emerged at the start of the first world war was the
    creation of permanent intelligence organizations. Previously, intelligence organizations
    were built up from scratch at the start of every conflict and dispanded soon after.
    This changed with the creation of permanent intelligence agencies during this period.
    Non only were intelligence agencies permanent, their percieved importance by goverments
    grew. With this increase in percieved importance the amount of resources made available
    to them grew as well.

    One such agency was "Room 40". Room 40 was created by the British and at the beginning
    none of its recruited members had experience in cryptology. The British also
    created another cryptoanalytic group known as "MI1b". Their purpose was breaking
    German cryptograms.

    There is a large difference between "ciphers" and "codes". These are independant
    cryptographic systems. The cryptoanalytic techniques for cracking them are also
    quite different.
    When it comes to breaking ciphers, frequency analysis, as well as various
    other techniques are employed.
    Codes require different techniques for breaking them. Codes rely mostly
    on guessing in order for a codebook to be incrementally built up.
    Another way of breaking codes simply involves recovering code books.
    Cryptoanalysis during this time involved processing a very large amount of
    ciphertext.
    Room 40 made significant progress when it was given access to a number of German
    naval codebooks.
    Another notable achievementof Room 40 was the breaking of the 13040 code.
    They did not break this code by finding codebooks. They broke this code
    by sifting through hundreds of coded messages.
    Room 40 ended up breaking superencipherments to often that the Germans
    ended up changing these keys every three months.
    In total, Room 40 decrypted more than 15 thousand German naval messages
    over the war.
    The Room 40 office also ended up employing hundreds of people towards the
    end of the war.
    One notable fact about Room 40 was that they were reticent in sharing their
    knowledge with their allies. The French for example, shared all of their
    information with the British despite Room 40 not sharing anything with them.

    Arguable, Room 40's greatest achievement during the first world war was when
    it decrypted a message about the Germans increasing submarine warfare.
    Room 40 also decrypted a German message that promised Mexico territory
    in the United States in exchange for their help. This message was significant
    because it would be important in America's decision to go to war.

    When it came to America's cryptological offices, it was quite far behind.
    Whereas the Europeans had had their Black Chambers for hundreds of years by now,
    by the start of the twentieth century the Americans had no offical office
    for cryptography. The cryptographic organizaitons that America had been using
    in previous wars were all temporary and were dispanded after these conflicts ended.

    This finally changed with the creation of the American "Army Signal School" in 1911.
    Notable figures that worked here included Joseph Mauborgne and Parker Hitt.
    In 1916, Hitt published the "Manual for the Solution of Military Ciphers".
    This manual detailed various kinds of ciphers as well as cryptoanalytic techniques
    that could be used in breaking them. Despite being somewhat outdated
    to techniques used in Europe, this book remained the go to handbook for American
    army cryptologists for a number of decades.

    When America joined the first world war, the American army had only a handful of
    cryptologists [22].
    One significant American cryptologists at this time was Herbert Yardley.
    He worked as a code clerk for the Washington State Department.
    Within the span of only a few hours, Yardley was able to break the encryption
    used in a letter from President Woodrow Wilson to one of his aids.
    Another one of Yardley's key achievements was the creation of a 100 page book
    that covered the codes and ciphers that the state used.

    Soon after the war started, Yardley was put in charge of the new cryptologic section
    named MI-8 or Section 8 [22]. In under a year, Section 8 grew from a handful of people,
    to being composed of various departments. By the time the war had ended
    it was composed of 165 people.
    Towards the end of the war, Yardley traveled to England and France to improve
    the sharing of intelligence between America and its allies.
    Again however, the British only shared very little with their American allies and did
    not even let Yardley visit Room 40.
    On his visit to France however, Yardley met with many French cryptanalysts.
    Though even the French did not share the diplomatic codes and ciphers they had recovered.
    By the end of the war, Yardley was part of the American delegation at
    the Versailles Peace Conference. MI-8 however, was already being scaled back
    in preparation for peace.

    Another major American intelligence unit during the war was the A.E.F. which
    operated in France. This unit was more geared towards tactical codes and ciphers.
    The American army in France had two main cryptologic branches.
    There was Military Intelligence and the Army Signal Corps.
    Military Intelligence also had a Radio Intelligence Section. Radio Intelligence
    was not only focused on code and cipher cryptanalysis, but also did traffic
    analysis and telephone interception. Futhermore, it also kept an eye on American
    communication security to check that security guidelines were properly followed.

    The Signal Corps on the other hand, had two code and cipher sections. These
    where to Code Compilation section and the radio interception section.
    The radio interception section listened in on German cryptograms sent by radio
    and then forwarded these on.
    The overall organizational structure of the American cryptological sections
    was overall quite similar to that of the British and French.

    Through the course of the war, both the Germans and the allies
    switched from ciphers to codes for use on the front lines. This was because
    ciphers proved to difficult to use in this scenario. These codes on the front
    lines were also called "trench codes".
    German "Satzbuch" codes were changed monthly which meant that the Americans
    had to work quickly breaking them.

    The trench codes for the American army were created by Code Compilation Section
    of the Signal Corps. The Code Compilation Section created these trench codes
    by starting with an outdated British trench code which they then improved upon.

    While developing their trench code, the Americans assigned Rives Childs to
    try to break the code. Given only 44 ciphertexts, Childs was able to find
    the entire cipher alphabet within five hours.
    This trench code was then completely discarded and the development of a new code began.
    A 2 part code was selected which, unlike the previous code, would use no superencipherment.
    The benefit of this 2 part code was that it was easy to encode and decode.

    The downside of this code was that enough ciphertext was recovered by the enemy,
    more and more parts of the codebook could be recovered.
    There was also the risk of a codebook being captured by the enemy.
    To mitigate these problems, a new codebook would be issued every two weeks.

    This series of codes would be dubbed the "River" codes and would prove
    to do their job well throughout the war. Every new variant of the code
    that would be issued was named after an American river. The entire
    code could fit in a pocket as it was only 47 pages long.
    Another nice feature the code had was the use of a font that was easy to read.

    The American 2nd Army recieved its own variation of the code dubbed the "Lake" series.
    This system was proved to be every effective throughout the war.
    Despite codebooks being captured by the Germans on three occasions,
    security was maintained due to new codes being issued within days.

    Another feature that was introduced to the system was the emergency code list.
    This code list was highly portable and easy to use.

    While trench codes were almost completely dominant among the armies of all sides,
    ciphers were still used to some degree.
    For the first two years of the war, the British made use of a field cipher
    for tactical communications.

    The Germans on the other hand also made use of a complex field cipher throughout
    the entire war but only for high level communication.

    The British used a cipher known as the "Playfair" cipher. The polymath
    Sir Charles Wheatstone invented this cipher back in 1854.
    It is called the "Playfair" cipher because it was heavily promoted
    by the Baron Lyon Playfair who was intent on having the British government
    use the cipher. A few decades after its invention, the Playfair was adopted
    as the goverment's official field cipher in the 1890s.
    Being first used in the Boer War, the cipher was used by the British in
    the first few years of the first world war.
    The Playfair is a variant of digraphic substition cipher.
    It works by encrypting two letters at a time and encrypting these
    plaintext digraphs into ciphertext digraphs.

    Where cryptanalysis is concerned, the most straighforward technique for
    breaking a Playfair cipher is freqency analysis.
    For frequency analysis to work here, a large volumn of ciphertext is needed.
    The frequency analysis will be performed on two letters at a time since
    the Playfair cipher uses digraphs.

    Arguably the most famous cipher of the first world war is the
    ADFGVX cipher. This cipher was created by the Germans who were
    preparing for a last desperate push to break the standstill.
    ADFGX was quite unique from anything the Germans had used
    throughout the war. ADFGX is a variant of fractionating cipher.
    Its called a fractionating cipher because it produces
    digraphs which are later broken in two.
    Encryption entailed three different steps.

    The only way the ADFGX cipher could be broken was by finding
    the sorted transposition key order.
    This was the task that Georges Painvin faced as he attempted
    to break this cipher in March 1918 just as the Germans were
    launching their offensive.
    One challenge that Painvin faced was a lack of ciphertext volume.

    Though this changed with the start of the offensive as
    more and more German communications were intercepted.

    Painvin's first step was find a bunch of messages that had
    similar beginnings and endings.
    Within three weeks he was already beginning to recover keys
    and break the cipher. He grew so proficient in breaking
    this cipher that on some days he was able to decrypt
    half of all messages.
    The Germans threw him a curve ball so to speak when they
    replaced the ADFGX cipher with a new variant known as ADFGVX.
    ADFGVX was different because a new row and column were added
    to the Polybius square.
    Painvin overcame the challenge of breaking this new cipher
    by working just one day and night.

    This feat made him the most famous cryptographer of World War 1.


    The first world war saw many advances in cryptography and cryptanalysis.
    One of the main lessons of the war was that the sheer volumn of
    messages as well as the risk of human error made manual encryption and
    decryption unreliable. These flaws could be overcome with the
    introduction of specialized cryptographic machines.
    The most basic cryptographic machines used at this time, cipher disks,
    had already been hundreds of years old by this point.

    Another one of these machines was the cipher cylinder.
    Five years after the first world war had ended, various kinds of
    machines had been created to generate polyalphabetic ciphertext.

    Arguably the most famous of all cipher machines that existed during this
    period is the Enigma. Rotors are used for both the encryption and decryption
    of polyalphabets. Arthur Scherbius first developed it in the 1920s.
    Various improvements were made to it over time. A few years
    after its creation if was officially being used by the German Army and Navy.

    One key attribute of the Enigma machine is that it is "self inverse".
    Self inverse cryptographic machines operate with the principle
    that encryption and decryption both use the same steps, albeit
    in a reverse order.
    This characteristic of the Enigma machine would later turn out
    to be its main weakness.


    Another serious weakness the Enigma machines had was that they
    had a reflector components which prevented any letter from being
    encrypted to itself.

    The Enigma machine would need two or three people to operate it.
    One inefficiency in its operation was due to the fact that letters
    were not printed out but instead had to be read through a set of lights.

    The Enigma machine was considered unbreakable by the Germans. This confidence
    in the machine's security was because of its great complexity as well
    as its high number of possible alphabets.

    The French, British, and Americans all bought Enigma machines and tried
    to break them.
    These early attempts at breaking the Enigma machine were met with failure.
    It was arguably the Poles however that had the greatest need to break the Enigma
    machine. They invested the most resources in breaking the Enigma machine
    because they knew that in the event of war, they would be among the first the Germans
    would attack. Poland had been part of the Prussian empire so the Germanys
    were expected to want this territory back.

    This led to the creation of Poland's own cipher bureau.
    With the creation of this bureau, mathematicians were also recruited
    and trained in cryptanalysis.
    One of these mathematicians was Marian Rejewski.
    In under a year, Rejewski had made enough progress to be able to decrypt
    certain Enigma messages.
    Rejewski's approach to break the cipher made use of the mathematical theory
    of permutations.
    He also had the advantage of having access to the day keys the Germans used
    with their Enigma machines.
    These advantages allowed the Polish to read decent amount of the German's
    Enigma messages.

    Then in 1938 the Germans made a large change in how the Enigma machine operated
    that made the techniques developed by the Poles so far useless.
    They accomplished this first by changing the indicator settings and later
    by adding two more rotors. These, among other improvement the Germans made,
    made the Enigma ciphers orders of magnitude harder to break.
    The Polish simply did not have the resources to overcome these new
    difficulties so they passed the knowledge they had gained so far to
    their allies.
    The Enigma machines were so sophisticated that attempting to break their
    encryption by hand would be impossible and the Allies would need to build specialized
    machines to break the encryption.

    Arguably the most notable figure in the story of the Enigma machines
    is Alan Turing. He graduated from Cambridge with first class honors in
    mathematics.
    He gained significant popularity in 1936 when he published the paper
    "On Computable Numbers". This paper was a response to a challenge
    set forth by the German mathematician David Hilbert.
    Turing answered Hilbert's decision problem through a theoretical abstract
    machine. This theoretical machine would later be known as the Turing Machine.

    In 1939, after German modifications to their Enigma machines thwarted
    Polish cryptoanalytic techniques, the British began looking
    for other cryptoanalytic techniques.
    In general, Turings approach for breaking the Enigma ciphers was
    to rule out as many impossible answers as possible.
    To this end he built a machine which he called "bombe".
    After some improvements and by using a large number of these "bombe" machines,
    the British were able to break the Enigma's daily keys in mere hours.

    \section{Random Number Generators}

    Random number generators, also known as RNGs, can be used for various different purposes [29].
    One of the main uses of RNGs is in simulations.
    Another use of RNGs is in sampling. The purpose of sampling is to choose
    "samples" at random which are a good representation of the true average.

    Random numbers are also generally used in programming. So called randomized algorithms
    make use of the indeterminism of random numbers.
    A very straighforward use of random numbers in programming
    is the generation of secret tokens that are used in authentication.
        
    \subsection{History of Random Number Generators}

    Another use of random numbers is when an unbiased decision needs to be made,
    like with a coin flip for example.

    When it comes to the actual meaning of randomness, a number of its own cannot be random.
    Instead we have a sequence of random numbers that follow a given distribution.
    This means that there should be no observable pattern between the numbers.
    The concept of uniformity that all numbers are equally likely to show up, without
    any favoritism making some numbers more common.
    
    In a truly random number generator, what numbers were generated in the past
    should not have any impact on the next number that is generated.

    Though sometimes pseudo random number generators are better for an application
    because they conform to the gambler's fallacy.
    The gambler's fallacy assumes that if a "random" result or set of random
    results has occurred, the next values that appear should skew towards being
    different to the random results observed so far.

    So for example, if a coin was flipped five times and the result was always heads,
    the gambler's fallacy would assume that there is a very high probability
    of the next result being tails. This is of course completely false in
    a truly random system.
    However in some cases it becomes desireable to have a pseudo random number system
    that exhibits this property.

    Why? Because sequences of a few items that skew greatly towards one average
    are less likely to appear.
    The final result of this is that the random number generator could be considered
    to be more "fair".
    This is no doublt the reasoning behind the creators of the computer game Dota 2
    implementing a system that has such "random" properties.

    Before machines were created to generate random numbers, several simple tools
    were used for this purpose.
    Such simple tools include dice, which have been used for thousands of years,
    as well as numbered balls that would be chosen at random.
    Cards were another way of generating random values.

    In 1927 a table of 40 thousand numbers was published by L. Tippett.
    Soon after this time however, devices were created that generated random
    values. In 1939 such a machine was created by Kendal and Babington-Smith
    to generate a table of 100 thousand random values.
    One very early computer, the Ferranti Mark I, had a dedicated instruction
    for generating random bits and storing them into an accumulator.
    The source of entropy used by this computer was resistance noise.
    Alan Turing had previously had an idea for such a system and this was
    its implementation.
    Another such device, named ERNIE, was used by the British lottery to
    generate the winning numbers.
    Once computers began to become popular, people started looking for ways
    to generate random values within software. Connecting a computer
    to a random generator like the ERNIE had the serious downside
    of not being able to reproduce a series of number deterministicly
    when using the same seed.

    Another downside of using such random number generating machines
    was that they could seriously malfunction in ways that were very difficult to
    detect.
    The practice of distributing lists of random numbers became popular again
    in the 90's were a billion random numbers could be stored on a CD.

    In 1946, von Neumann came up with an algorithm for generating random
    numbers completely using arithmetic operations on a computer.
    Of course, such a sequence that was generated deterministicly through
    arithmetic operations could be argued to not actually be random.

    These numbers would have the appearance of being random however.
    This early software RNG used a middle square to generate subsequent
    outputs from a seed.
    Despite the philosophical concept of these numbers lacking true randomness,
    it can still be argued that random number generation with
    arithmetic is still practical for most applications.

    Von Neumann's initial approach using the middle square proved to have
    some serious flaws though. The problem was that this RNG could
    reach states where the RNG would repeat the same short sequence of
    numbers again and again. For instance if zero was reached by the RNG
    it would stop generating different values entirely and stay at zero.

    Some improvements to the middle square method were made and this allowed
    the generator to output nearly a million values before it became trapped
    in an undeseriable cyclic state.

    In 1959, another type of RNG known as "Algorithm K" was created [29].
    Despite looking promsing from the outset, Algorithm K, upon closing inspection,
    would be shown to have serious flaws.

    It has a very small period of only three thousand numbers.
    Algorithm K proved that RNGs need to be crafted very deliberately and cannot simply
    created by adding various arbitrary instructions and hoping for the best.
    Greater theoretical understanding was needed.
        
    \subsection{Reliable RNG Algorithms}

    One of the most reliable and widely adopted RNG is the Linear Congruential Method [29].
    It was first introduced by Lehmer in 1949 and has four internal values
    that can be adjusted as needed. These include the modulus, multiplier, increment
    and starting value (also known as the seed). These numbers need
    to be methodically selected in order for a desirable output or random numbers to appear.
    A wrong selection of these number can cause the output to have a very small period,
    repeating only a handful of numbers again and again.
    There are actually two different variants of this generator, the multiplicative congruential
    method was well as the mixed congruential method. The difference
    between these variations is that the former uses an increment value equal to zero while
    the latter uses a nonzero increment.

    
    
    \section{Related Work}

    Todo - add this

    THIS COMES NEAR THE END

    \break
    \section*{References}

    \begin{enumerate}

    \item Chandrasekaran, Shrutisagar, and Abbes Amira. "High performance FPGA implementation of the Mersenne Twister." 4th IEEE International Symposium on Electronic Design, Test and Applications (delta 2008). IEEE, 2008.

    \end{enumerate}
    
\end{document}
