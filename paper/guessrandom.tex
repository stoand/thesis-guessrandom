\documentclass{article}
\usepackage[utf8]{inputenc}
\usepackage[english]{babel}
\usepackage[document]{ragged2e}
\usepackage{multicol}
\usepackage[margin=0.5in]{geometry}

\begin{document}

    \begin{center}
    \end{center}
    
    \addvspace{20mm}
        
    \begin{center}
        \huge Novel use of a Functional HDL to Simplify Development of an RNG Brute-Force Algorithm on an FGPA
    \end{center}
    
    \begin{center}
    \end{center}
       
    \begin{center}
        \large Andreas Stocker
    \end{center}
    
    \begin{center}
        \small \emph {University of Nicosia}
    \end{center}

    \addvspace{15mm}

    \begin{multicols}{2}

    % Abstract

    In this paper we go over the implementation details of implementing an algorithm that brute forces
    the state of a random number generator. We do so using a hardware description language whose syntax
    resembles that of the Haskell language.

    Throughout this paper we assess the practicality of using this functional-style approach in
    designing calculation-heavy algorithms for an FPGA.
        

    \section{Introduction}

    todo intro
    
    \section{Related Work}

    Todo - add this

    THIS COMES NEAR THE END

    
    \end{multicols}

    \break
    \section*{References}

    \begin{enumerate}

    \item S.  Abiteboul  and  R.  Hull,  Data  functions,  Datalog  and  negation,in``Proceedings  of  the  ACM-SIGMOD  International  Conference  onManagement of Data,'' pp. 143153, 1988.

    \end{enumerate}
    
\end{document}
